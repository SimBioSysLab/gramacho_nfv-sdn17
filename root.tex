%%%%%%%%%%%%%%%%%%%%%%%%%%%%%%%%%%%%%%%%%%%%%%%%%%%%%%%%%%%%%%%%%%%%%%%%%%%%%%%%
%2345678901234567890123456789012345678901234567890123456789012345678901234567890
%        1         2         3         4         5         6         7         8

\documentclass[letterpaper, 10 pt, conference]{ieeeconf}  % Comment this line out if you need a4paper

%\documentclass[a4paper, 10pt, conference]{ieeeconf}      % Use this line for a4 paper

\IEEEoverridecommandlockouts                              % This command is only needed if 
                                                          % you want to use the \thanks command

\overrideIEEEmargins                                      % Needed to meet printer requirements.

% See the \addtolength command later in the file to balance the column lengths
% on the last page of the document

\usepackage{epsfig,endnotes,float}
\usepackage{ifthen,graphicx,verbatim}
\usepackage{xspace,balance}
\usepackage{colortbl,booktabs}
\usepackage[obeyspaces,hyphens]{url}
\usepackage{amsmath,cite}
\usepackage{inconsolata,array}
\usepackage{xspace,balance}
\usepackage{acronym}
\usepackage{subfig}

\newcommand{\ie}{\emph{i.e.},\xspace}
\newcommand{\eg}{\emph{e.g.},\xspace}
\newcommand{\etal}{\emph{et al.\ }}
\newcommand{\fixme}{\large\textbf{FIXMEFIXMEFIXME!}\normalsize}

\newenvironment{itemtight}{
 \begin{list}{$\bullet$\hfill}
  {\setlength{\parsep}{0ex}\setlength{\itemsep}{0ex}
   \setlength{\labelwidth}{0.10in}\setlength{\labelsep}{0.05in}
   \setlength{\leftmargin}{0.15in}\setlength{\rightmargin}{0.0in}
  }
}{
 \end{list}
}

\newcounter{enumtight}
\newenvironment{enumtight}{
 \begin{list}{\arabic{enumtight}.\hfill}
  {\usecounter{enumtight}
   \setlength{\parsep}{0ex}\setlength{\itemsep}{0ex}
   \setlength{\labelwidth}{0.10in}\setlength{\labelsep}{0.05in}
   \setlength{\leftmargin}{0.15in}\setlength{\rightmargin}{0.0in}
  }
}{
 \end{list}
}

\title{\LARGE \bf
Architecting SDNs for Wireless Mesh Networks \fixme - Working title: we need a name for this system.
}


\author{Sergio Gramacho and Avani Wildani% <-this % stops a space
\thanks{*This work is sponsored by the CAPES foundation and Emory PERS}% <-this % stops a space
\thanks{Contact authors at: Department of Math and Computer Science,
        Emory University, Atlanta, GA.
        {\tt\small \{sgramac,avani\}@mathcs.emory.edu}}%
}


\begin{document}



\maketitle
\thispagestyle{empty}
\pagestyle{empty}


%%%%%%%%%%%%%%%%%%%%%%%%%%%%%%%%%%%%%%%%%%%%%%%%%%%%%%%%%%%%%%%%%%%%%%%%%%%%%%%%
\begin{abstract}

\begin{itemize}
    \item Connectivity in the 3rd world is terrible
    \items WMNs help this, and trying out different SDN-WMN architectures is critical to building better systems at scale
    \item Simulating WMNs at scale is fraught with tricky details
    \item We solved the tricky details and increased the complexity of the system we could simulate from X to Y. (\fixme need a definition of complexity) 
\end{itemize}
\end{abstract}


%%%%%%%%%%%%%%%%%%%%%%%%%%%%%%%%%%%%%%%%%%%%%%%%%%%%%%%%%%%%%%%%%%%%%%%%%%%%%%%%
\section{Introduction}

\begin{itemize}
    \item Motivate problem: access to Internet
    \item Briefly summarize approach and key improvements
\end{itemize}

\section{Background}

\subsection{Other simulation frameworks}
\subsection{Real deployments and their limitations}

\section{Architecture}

\subsection{ns3}
\subsection{AI}


\section{Scalability Improvements}

\begin{itemize}
    \item Discuss GCE implementation
    \item Go through, in detail, the different problems and solutions you implemented to up the number of nodes.
\end{itemize}

\begin{itemize}
    \item GRAPH: # nodes vs. runtime
    \item GRAPH: Runtime change with distributed vs. SDN models
    \item GRAPH: 
\end{itemize}

\section{Discussion}
\begin{itemize}
    \item Talk about why the architecture choices you made worked.
    \item Discuss what you'd do to parallelize across multiple machines
\end{itemize}
\section{Conclusions}
\end{document}
